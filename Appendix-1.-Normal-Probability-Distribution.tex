% Options for packages loaded elsewhere
\PassOptionsToPackage{unicode}{hyperref}
\PassOptionsToPackage{hyphens}{url}
\PassOptionsToPackage{dvipsnames,svgnames,x11names}{xcolor}
\documentclass[
]{article}
\usepackage{xcolor}
\usepackage[margin=1in]{geometry}
\usepackage{amsmath,amssymb}
\setcounter{secnumdepth}{5}
\usepackage{iftex}
\ifPDFTeX
  \usepackage[T1]{fontenc}
  \usepackage[utf8]{inputenc}
  \usepackage{textcomp} % provide euro and other symbols
\else % if luatex or xetex
  \usepackage{unicode-math} % this also loads fontspec
  \defaultfontfeatures{Scale=MatchLowercase}
  \defaultfontfeatures[\rmfamily]{Ligatures=TeX,Scale=1}
\fi
\usepackage{lmodern}
\ifPDFTeX\else
  % xetex/luatex font selection
\fi
% Use upquote if available, for straight quotes in verbatim environments
\IfFileExists{upquote.sty}{\usepackage{upquote}}{}
\IfFileExists{microtype.sty}{% use microtype if available
  \usepackage[]{microtype}
  \UseMicrotypeSet[protrusion]{basicmath} % disable protrusion for tt fonts
}{}
\makeatletter
\@ifundefined{KOMAClassName}{% if non-KOMA class
  \IfFileExists{parskip.sty}{%
    \usepackage{parskip}
  }{% else
    \setlength{\parindent}{0pt}
    \setlength{\parskip}{6pt plus 2pt minus 1pt}}
}{% if KOMA class
  \KOMAoptions{parskip=half}}
\makeatother
\usepackage{color}
\usepackage{fancyvrb}
\newcommand{\VerbBar}{|}
\newcommand{\VERB}{\Verb[commandchars=\\\{\}]}
\DefineVerbatimEnvironment{Highlighting}{Verbatim}{commandchars=\\\{\}}
% Add ',fontsize=\small' for more characters per line
\usepackage{framed}
\definecolor{shadecolor}{RGB}{248,248,248}
\newenvironment{Shaded}{\begin{snugshade}}{\end{snugshade}}
\newcommand{\AlertTok}[1]{\textcolor[rgb]{0.94,0.16,0.16}{#1}}
\newcommand{\AnnotationTok}[1]{\textcolor[rgb]{0.56,0.35,0.01}{\textbf{\textit{#1}}}}
\newcommand{\AttributeTok}[1]{\textcolor[rgb]{0.13,0.29,0.53}{#1}}
\newcommand{\BaseNTok}[1]{\textcolor[rgb]{0.00,0.00,0.81}{#1}}
\newcommand{\BuiltInTok}[1]{#1}
\newcommand{\CharTok}[1]{\textcolor[rgb]{0.31,0.60,0.02}{#1}}
\newcommand{\CommentTok}[1]{\textcolor[rgb]{0.56,0.35,0.01}{\textit{#1}}}
\newcommand{\CommentVarTok}[1]{\textcolor[rgb]{0.56,0.35,0.01}{\textbf{\textit{#1}}}}
\newcommand{\ConstantTok}[1]{\textcolor[rgb]{0.56,0.35,0.01}{#1}}
\newcommand{\ControlFlowTok}[1]{\textcolor[rgb]{0.13,0.29,0.53}{\textbf{#1}}}
\newcommand{\DataTypeTok}[1]{\textcolor[rgb]{0.13,0.29,0.53}{#1}}
\newcommand{\DecValTok}[1]{\textcolor[rgb]{0.00,0.00,0.81}{#1}}
\newcommand{\DocumentationTok}[1]{\textcolor[rgb]{0.56,0.35,0.01}{\textbf{\textit{#1}}}}
\newcommand{\ErrorTok}[1]{\textcolor[rgb]{0.64,0.00,0.00}{\textbf{#1}}}
\newcommand{\ExtensionTok}[1]{#1}
\newcommand{\FloatTok}[1]{\textcolor[rgb]{0.00,0.00,0.81}{#1}}
\newcommand{\FunctionTok}[1]{\textcolor[rgb]{0.13,0.29,0.53}{\textbf{#1}}}
\newcommand{\ImportTok}[1]{#1}
\newcommand{\InformationTok}[1]{\textcolor[rgb]{0.56,0.35,0.01}{\textbf{\textit{#1}}}}
\newcommand{\KeywordTok}[1]{\textcolor[rgb]{0.13,0.29,0.53}{\textbf{#1}}}
\newcommand{\NormalTok}[1]{#1}
\newcommand{\OperatorTok}[1]{\textcolor[rgb]{0.81,0.36,0.00}{\textbf{#1}}}
\newcommand{\OtherTok}[1]{\textcolor[rgb]{0.56,0.35,0.01}{#1}}
\newcommand{\PreprocessorTok}[1]{\textcolor[rgb]{0.56,0.35,0.01}{\textit{#1}}}
\newcommand{\RegionMarkerTok}[1]{#1}
\newcommand{\SpecialCharTok}[1]{\textcolor[rgb]{0.81,0.36,0.00}{\textbf{#1}}}
\newcommand{\SpecialStringTok}[1]{\textcolor[rgb]{0.31,0.60,0.02}{#1}}
\newcommand{\StringTok}[1]{\textcolor[rgb]{0.31,0.60,0.02}{#1}}
\newcommand{\VariableTok}[1]{\textcolor[rgb]{0.00,0.00,0.00}{#1}}
\newcommand{\VerbatimStringTok}[1]{\textcolor[rgb]{0.31,0.60,0.02}{#1}}
\newcommand{\WarningTok}[1]{\textcolor[rgb]{0.56,0.35,0.01}{\textbf{\textit{#1}}}}
\usepackage{graphicx}
\makeatletter
\newsavebox\pandoc@box
\newcommand*\pandocbounded[1]{% scales image to fit in text height/width
  \sbox\pandoc@box{#1}%
  \Gscale@div\@tempa{\textheight}{\dimexpr\ht\pandoc@box+\dp\pandoc@box\relax}%
  \Gscale@div\@tempb{\linewidth}{\wd\pandoc@box}%
  \ifdim\@tempb\p@<\@tempa\p@\let\@tempa\@tempb\fi% select the smaller of both
  \ifdim\@tempa\p@<\p@\scalebox{\@tempa}{\usebox\pandoc@box}%
  \else\usebox{\pandoc@box}%
  \fi%
}
% Set default figure placement to htbp
\def\fps@figure{htbp}
\makeatother
\setlength{\emergencystretch}{3em} % prevent overfull lines
\providecommand{\tightlist}{%
  \setlength{\itemsep}{0pt}\setlength{\parskip}{0pt}}
\usepackage{bookmark}
\IfFileExists{xurl.sty}{\usepackage{xurl}}{} % add URL line breaks if available
\urlstyle{same}
\hypersetup{
  pdftitle={Normal Probability Distribution},
  pdfauthor={Thomas Bryan Smith  },
  colorlinks=true,
  linkcolor={magenta},
  filecolor={Maroon},
  citecolor={Blue},
  urlcolor={magenta},
  pdfcreator={LaTeX via pandoc}}

\title{Normal Probability Distribution}
\usepackage{etoolbox}
\makeatletter
\providecommand{\subtitle}[1]{% add subtitle to \maketitle
  \apptocmd{\@title}{\par {\large #1 \par}}{}{}
}
\makeatother
\subtitle{CJ 702: Advanced Criminal Justice Statistics}
\author{Thomas Bryan Smith\footnote{University of Mississippi,
  \href{mailto:tbsmit10@olemiss.edu}{\nolinkurl{tbsmit10@olemiss.edu}}}}
\date{February 03, 2025}

\begin{document}
\maketitle

{
\hypersetup{linkcolor=}
\setcounter{tocdepth}{2}
\tableofcontents
}
\section{Load the USArrest data}\label{load-the-usarrest-data}

First, let's load in the built-in USArrest data, and take a look at it.

\begin{Shaded}
\begin{Highlighting}[]
\FunctionTok{data}\NormalTok{(USArrests)}

\FunctionTok{head}\NormalTok{(USArrests, }\DecValTok{25}\NormalTok{)}
\end{Highlighting}
\end{Shaded}

\begin{verbatim}
##               Murder Assault UrbanPop Rape
## Alabama         13.2     236       58 21.2
## Alaska          10.0     263       48 44.5
## Arizona          8.1     294       80 31.0
## Arkansas         8.8     190       50 19.5
## California       9.0     276       91 40.6
## Colorado         7.9     204       78 38.7
## Connecticut      3.3     110       77 11.1
## Delaware         5.9     238       72 15.8
## Florida         15.4     335       80 31.9
## Georgia         17.4     211       60 25.8
## Hawaii           5.3      46       83 20.2
## Idaho            2.6     120       54 14.2
## Illinois        10.4     249       83 24.0
## Indiana          7.2     113       65 21.0
## Iowa             2.2      56       57 11.3
## Kansas           6.0     115       66 18.0
## Kentucky         9.7     109       52 16.3
## Louisiana       15.4     249       66 22.2
## Maine            2.1      83       51  7.8
## Maryland        11.3     300       67 27.8
## Massachusetts    4.4     149       85 16.3
## Michigan        12.1     255       74 35.1
## Minnesota        2.7      72       66 14.9
## Mississippi     16.1     259       44 17.1
## Missouri         9.0     178       70 28.2
\end{verbatim}

\section{\texorpdfstring{Viewing the \emph{frequency distribution} for
the \textbf{Assault}
variable}{Viewing the frequency distribution for the Assault variable}}\label{viewing-the-frequency-distribution-for-the-assault-variable}

Now, let's visualize the \emph{frequency distribution} for the Assault
variable. Remember, the \emph{observed frequency distribution} is not
the same as the \emph{probability distribution}.

\begin{Shaded}
\begin{Highlighting}[]
\FunctionTok{ggplot}\NormalTok{(USArrests, }\FunctionTok{aes}\NormalTok{(}\AttributeTok{x =}\NormalTok{ Assault)) }\SpecialCharTok{+}
  \FunctionTok{geom\_histogram}\NormalTok{(}\AttributeTok{bins =} \DecValTok{30}\NormalTok{)}
\end{Highlighting}
\end{Shaded}

\pandocbounded{\includegraphics[keepaspectratio]{Appendix-1.-Normal-Probability-Distribution_files/figure-latex/freq-1.pdf}}

\section{\texorpdfstring{The \textbf{normal} \emph{probability
distribution}}{The normal probability distribution}}\label{the-normal-probability-distribution}

The normal probability distribution is typically indicated with the
following expression:

\[ N(\mu , \sigma^{2}) \]

Quite literally just saying that you are working with a normal
distribution with a given mean, \(\mu\), and standard deviation,
\(\sigma\).

We could visually generate this probability distribution using the
following probability density function:

\[ f(x) = \frac{1}{\sigma \sqrt{2 \pi}}e^{- \frac{1}{2}(\frac{x - \mu}{\sigma})^{2}} \]

This can be used to generate a normal distribution that represents all
theoretically possible values of a given normally distributed continuous
variable (where your histogram is beholden to the observations that
\emph{actually exist}). As ever, \(\mu\) is the mean of your variable,
\(\sigma\) is the standard deviation of your variable,

\begin{Shaded}
\begin{Highlighting}[]
\CommentTok{\# Find the mean:}
\NormalTok{mu }\OtherTok{\textless{}{-}}\NormalTok{ USArrests}\SpecialCharTok{$}\NormalTok{Assault }\SpecialCharTok{\%\textgreater{}\%} \FunctionTok{mean}\NormalTok{(}\AttributeTok{na.rm =} \ConstantTok{TRUE}\NormalTok{)}

\CommentTok{\# Find the standard deviation:}
\NormalTok{sigma }\OtherTok{\textless{}{-}}\NormalTok{ USArrests}\SpecialCharTok{$}\NormalTok{Assault }\SpecialCharTok{\%\textgreater{}\%} \FunctionTok{sd}\NormalTok{(}\AttributeTok{na.rm =} \ConstantTok{TRUE}\NormalTok{)}

\CommentTok{\# Generate a sequence of all possible values for x:}
\NormalTok{min }\OtherTok{\textless{}{-}}\NormalTok{ USArrests}\SpecialCharTok{$}\NormalTok{Assault }\SpecialCharTok{\%\textgreater{}\%} \FunctionTok{min}\NormalTok{(}\AttributeTok{na.rm =} \ConstantTok{TRUE}\NormalTok{)}
\NormalTok{max }\OtherTok{\textless{}{-}}\NormalTok{ USArrests}\SpecialCharTok{$}\NormalTok{Assault }\SpecialCharTok{\%\textgreater{}\%} \FunctionTok{max}\NormalTok{(}\AttributeTok{na.rm =} \ConstantTok{TRUE}\NormalTok{)}

\NormalTok{x }\OtherTok{\textless{}{-}} \FunctionTok{seq}\NormalTok{(min, max, }\AttributeTok{by =} \FloatTok{0.1}\NormalTok{)}

\CommentTok{\# Insert these values into the normal probability density function:}
\NormalTok{npd }\OtherTok{\textless{}{-}}\NormalTok{ (}\DecValTok{1} \SpecialCharTok{/}\NormalTok{ (sigma }\SpecialCharTok{*} \FunctionTok{sqrt}\NormalTok{(}\DecValTok{2} \SpecialCharTok{*}\NormalTok{ pi))) }\SpecialCharTok{*} \FunctionTok{exp}\NormalTok{(}\DecValTok{1}\NormalTok{)}\SpecialCharTok{\^{}}\NormalTok{(}\SpecialCharTok{{-}}\NormalTok{(x }\SpecialCharTok{{-}}\NormalTok{ mu)}\SpecialCharTok{\^{}}\DecValTok{2} \SpecialCharTok{/}\NormalTok{ (}\DecValTok{2} \SpecialCharTok{*}\NormalTok{ sigma}\SpecialCharTok{\^{}}\DecValTok{2}\NormalTok{))}

\FunctionTok{plot}\NormalTok{(npd, x)}
\end{Highlighting}
\end{Shaded}

\pandocbounded{\includegraphics[keepaspectratio]{Appendix-1.-Normal-Probability-Distribution_files/figure-latex/prob-1.pdf}}

\end{document}
